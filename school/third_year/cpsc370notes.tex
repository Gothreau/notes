% Created 2016-12-05 Mon 16:09
\documentclass[11pt]{article}
\usepackage[utf8]{inputenc}
\usepackage[T1]{fontenc}
\usepackage{fixltx2e}
\usepackage{graphicx}
\usepackage{longtable}
\usepackage{float}
\usepackage{wrapfig}
\usepackage{rotating}
\usepackage[normalem]{ulem}
\usepackage{amsmath}
\usepackage{textcomp}
\usepackage{marvosym}
\usepackage{wasysym}
\usepackage{amssymb}
\usepackage{hyperref}
\tolerance=1000
\author{Barry }
\date{\today}
\title{cpsc370notes}
\hypersetup{
  pdfkeywords={},
  pdfsubject={},
  pdfcreator={Emacs 25.1.1 (Org mode 8.2.10)}}
\begin{document}

\maketitle
\tableofcontents

\section{lecture 2}
\label{sec-1}
\subsection{relations}
\label{sec-1-1}
\begin{itemize}
\item a relation r from A to B is a subset of A x B
\begin{itemize}
\item or sometimes (A, B, r) where r is a subset of A X B
\end{itemize}
\item for a function the set A is called the domain and B is the co-domain
\end{itemize}

\subsection{functions}
\label{sec-1-2}
\begin{itemize}
\item a function f from A to B ( f: A->B) satisfies the two rules
\begin{itemize}
\item (i) must answer every question
\begin{itemize}
\item for all a in A there exists a b in B such that (a,b) is in f
\end{itemize}
\item (ii) one answer per question
\begin{itemize}
\item for any a in A and any b1,b2 in B if (a,b1) and (a,b2) are both in f then b1=b2
\end{itemize}
\end{itemize}
\item a partial function f from A to B is a relation that satisfies ii
\begin{itemize}
\item relevant to cs because programming functions dont neccesarily have an answer for every value
\end{itemize}
\item given finite sets A and B how many functions are there from A to B
\begin{itemize}
\item |B$^{\text{A|}}$ = |B|$^{\text{|A|}}$
\end{itemize}
\item how many partial functions are there from A to B
\begin{itemize}
\item add an unknown to B then the answer is (|B U \{?\}|$^{\text{|A|}}$)
\begin{itemize}
\item like null
\end{itemize}
\end{itemize}
\item how do we cope with multiple arguments in a function
\begin{itemize}
\item mathematics treats multi-arg functions as functions from a cross-product
\begin{itemize}
\item fourier: (Integers x Real numbers) -> Real numbers
\end{itemize}
\end{itemize}
\end{itemize}

\section{lecture 3}
\label{sec-2}
\begin{itemize}
\item currying is functions that return functions and is a way to deal with multiple argument functions
\begin{itemize}
\item a function from AxB to C can be though of as a function from a to a function of B to C
\item named after a logistian Haskell Curry
\begin{itemize}
\item invented by Schonfinkel (o has an umlout)
\end{itemize}
\end{itemize}
\item two charateristics of a programming language that suggest it's functional are
\begin{itemize}
\item anonymous functions
\item the ability to return functions from functions
\end{itemize}
\item Haskell syntax
\begin{itemize}
\item g (n,x) = sin (fromRational n * x)
\begin{itemize}
\item can also include types by putting the following first g :(two colons) Int * Double -> Double
\item curried version: f n x = sin (fromRational n * x)
\begin{itemize}
\item f3 = f(3)
\item f3(2.75)
\end{itemize}
\end{itemize}
\item -- is a comment in Haskell (minus minus)
\end{itemize}
\item Scheme syntax
\begin{itemize}
\item (define (g n x) (sin (* n x)))
\begin{itemize}
\item (g 3 4.5)
\end{itemize}
\end{itemize}
\end{itemize}

\section{lecture 4}
\label{sec-3}
syntax notes for haskel
\begin{itemize}
\item funtion application is juxtaposition (placing next to eah other)
\begin{itemize}
\item f (x) = x + 1
\begin{itemize}
\item f (16)
\item or 'f 16' both are valid
\end{itemize}
\item juxtaposition binds very tightly
\item there is an operator '\$' which has low precedence and binds right to left
\end{itemize}
\item ordinary names in haskell consist of letters, numbers and \_, with the usual rules
\begin{itemize}
\item functions \& variables have names that start with lower case letters
\item operator names
\begin{itemize}
\item consist of !\#\$\%\&+\_./*<=>?@$\backslash$$^{\text{|}}$\textasciitilde{} and :
\end{itemize}
\end{itemize}
\item use function in an infix way by placing them in backticks
\begin{itemize}
\item 3 `f` 1
\end{itemize}
\item operators can be used in a prefix way by surrounding with ()
\begin{itemize}
\item + 5 7
\end{itemize}
\item syntactic sugar
\begin{itemize}
\item makes a language sweeter
\item adds usability to a language without adding functionality
\end{itemize}
\end{itemize}
\subsection{{\bfseries\sffamily DONE} read learn you a haskell for great good chapter 2}
\label{sec-3-1}
\subsection{{\bfseries\sffamily DONE} question 1}
\label{sec-3-2}
email
\subsection{{\bfseries\sffamily DONE} question 2}
\label{sec-3-3}
explain how to write a function to return a*(x$^{\text{2}}$)+b*x+c
\subsection{{\bfseries\sffamily DONE} question 3}
\label{sec-3-4}
write a curryed function to return the function x-> ax$^{\text{2}}$ + bx + c and execute it in 
\begin{itemize}
\item java
\item haskell
\item racket
\end{itemize}
\section{lecture 5}
\label{sec-4}
\begin{itemize}
\item basic list in functional languages
\begin{itemize}
\item lists are singly linked
\item frequently immutable
\item made from cons cells and the empty list
\begin{itemize}
\item cons cells are a pinter to a piece of data and the rest of the list
\end{itemize}
\end{itemize}
\item static typing
\begin{itemize}
\item a language is statically typed if you can work out the type of every expression at compile time
\item a language is statically typed if there is an algorithm to determine the type of the expression
\end{itemize}
\end{itemize}

\subsection{{\bfseries\sffamily DONE} question 4}
\label{sec-4-1}
How many partial functions are there from \{ Scissors, Paper, Rock, Spock, Lizard \} 2 to \{ Win, Lose \} ?
How many of these are fair and interesting?
\subsection{{\bfseries\sffamily DONE} read chapters 3+4}
\label{sec-4-2}
\section{lecture 6}
\label{sec-5}
in functional languages functions are first class entities, first class means something like fully integrated into the
language
\begin{itemize}
\item first class
\begin{itemize}
\item create literals and/or use without naming
\item use them as arguments or return types
\item store them in arrays and other data structures
\item read/write to/from files
\item compare for equality
\item order them
\item look inside their contents
\end{itemize}
\item haskel function to determine length of a list
\end{itemize}
length [] = 0
length (\_:xs) = 1 + length xs
\begin{itemize}
\item fib (not great)
\end{itemize}
fib n = if n == 0
        then 0
        else if n == 1
             then 1
             else fib(n-1) + fib(n-2)
\begin{itemize}
\item fib good
\end{itemize}
fib 0 = 0
fib 1 = 1
fib n = fib(n-1) + fib(n-2)
\begin{itemize}
\item another fib
\end{itemize}
fib n
\begin{center}
\begin{tabular}{l}
n == 0    = 0\\
n == 1    = 1\\
n < 0     = undefined\\
otherwise = fib(n-1) + fib(n-2)\\
\end{tabular}
\end{center}
\section{lecture 7}
\label{sec-6}
c(n) = 3n+1 if n is odd, n/2 if n is even
\section{lecture 8}
\label{sec-7}
haskell merge sort:
mergeSort [] = []
mergeSort [x] = [x]
mergeSort xs = merge front back where
     front = mergeSort \$ m `take` xs 
     back = mergeSort \$ m `drop` xs
     m = length xs `div` 2
     merge [] ys = ys
     merge xs [] = xs
     merge (x:xs) (y:ys) =
         if y<x
         then y : merge (x:xs) ys
         else x : merge xs (y:ys)

\begin{itemize}
\item to use recursion
\begin{enumerate}
\item I need a well ordered domain (for any subset there is a lease element)
\item I need to know the answers for base cases
\item I need to know how to link the answer of a general problem to the answer of a smaller problem
\end{enumerate}
\item 
\end{itemize}
\section{lecture 9}
\label{sec-8}
\begin{itemize}
\item homeworks 1-4 are due oct 4
\item spent the whole class working on the same problem from lecture 7, started in haskell
\end{itemize}
\section{lecture 10}
\label{sec-9}
how to write zip in haskel (takes two lists and returns a list of pairs)

zip \_ [] = []
zip [] \_ = []
zip (x:xs) (y:ys) = (x, y) : zip xs ys

\begin{itemize}
\item strict vs lazy languages
\begin{itemize}
\item a language uses strict evaluation (is strict) if it always evaluates the arguments of a function 
before evaluating the body of the function
\item a language is lazy if it  only evaluates function arguments when they are used
\end{itemize}
\end{itemize}

consider:
pi1 x y = x
badloop x = badloopx
pi1 true (badloop false)

in a lazy language we would get true in a strict language we get stuck forever

sq x = x * x
sq (sq (sq (sq (1+1) )))
-- a more haskelly way to do this would be
sq . sq .sq . sq \$ 1 + 1
-- . is function composition

\begin{itemize}
\item a lazy language would multiply 1+1 many times because it would try to evaluate things one layer at a 
time, a strongly typed language would multiply 1+1 once.
\begin{itemize}
\item no real lazy language would actually do this, a well designed interpreter/compiler will make sure 
arguments are computed 0 or 1 times
\end{itemize}

\item scheme is strict
\item haskell is lazy
\item java is strict
\item most languages are strict
\item smart laziness involves only evaluating an argument at most once
\end{itemize}

(define (square x) (* x x))
(define (side-effect)
    (begin
    (displayln "Hi")
    2)) ;; returns 2 with the side effect of printing to the screen
(square (square side-effect))

\begin{itemize}
\item in a strict language Hi is printed once
\item in a lazy language square will fracture multiple times an side effect would be called multiple times
\begin{itemize}
\item sq (sq side-effect)
\begin{itemize}
\item sq (side-effect * side-effect)
\begin{itemize}
\item (side-effect * side-effect) * (side-effect * side-effect)
\end{itemize}
\end{itemize}
\end{itemize}
\item laziness only makes sense in pure (non-side-effecting) languages
\begin{itemize}
\item haskell is pure
\end{itemize}
\item a "thunk" is either
\begin{itemize}
\item a technique for creating a delayed computation (our use)
\item a way to implement run time method selection especially in multiple inheritance languages
\end{itemize}
\end{itemize}


\begin{itemize}
\item a thunk is one of two things
\begin{itemize}
\item a way of delaying an operation (our use)
\begin{itemize}
\item replace an expression with something that stands for it so we don't have to recompute a value
\end{itemize}
\item a way of picking a method in a multi-inheritance language
\end{itemize}
\end{itemize}

define f (λ () (+ 2 3)) )
Racket boxes
box
unbox
setbox!

(define g (box 5))
(setbox! g "cat")

;; a box is like a one element array

g -> [] -> 5
\begin{center}
\begin{tabular}{}
\\
\end{tabular}
\end{center}
  v
"cat"

(define (delay f)
    (box (cons \#false f))
)

(define (force g)
  (let ([contents (unbox g)])
  (if (car contents)
      (cdr contents)
      (let ([result ((cdr contents))])
           (setbox! g (cons \#true result))
           result)
  )
  )
)


\rule{\linewidth}{0.5pt}
fib$_{\text{nums}}$ = 0:1:zipWith (+) fib$_{\text{nums}}$ (drop 1 fib$_{\text{nums}}$)

this is fine as long as we don't try to compute all of them
take 1000 fib$_{\text{nums}}$
would work fine

\subsection{{\bfseries\sffamily DONE} miderm}
\label{sec-9-1}
\begin{itemize}
\item summarize laziness
\item summarize strictness
\item example where it makes a difference
\item explain what a thunk is and why we use it
\begin{itemize}
\item why do we have thunks in lazy languages
\begin{itemize}
\item so we don't have to evaluate arguments multiple times
\end{itemize}
\end{itemize}
\item which of the two languages we're studying are lazy
\item which is statically typed
\item why are functional languages tricky
\item recursion
\end{itemize}

\subsection{{\bfseries\sffamily DONE} curry/uncurry functions}
\label{sec-9-2}
f x y = x+y
ff (x,y) = x+y

\begin{itemize}
\item in haskell write an uncurry function so that (uncurry f) is the samme as ff.
\item in racket write a curry function
\end{itemize}

\section{lecture 11}
\label{sec-10}
\begin{itemize}
\item Environments
\begin{itemize}
\item abstractly an environment is a dictionary mapping identifiers to their meanings at some particular time
\end{itemize}
\end{itemize}

public static int fact(int n) \{
  if (n == 0)
     return 1;
  else return n * fact(n-1);
\}

stack:
\begin{center}
\begin{tabular}{ll}
fact & \footnotemarkn\\
fact & \footnotemarkn & <- stack frame\\
fact & \footnotemarkn\\
. & \\
. & \\
. & \\
main & \\
\end{tabular}
\end{center}\footnotetext[1]{DEFINITION NOT FOUND.}\footnotetext[2]{DEFINITION NOT FOUND.}\footnotetext[3]{DEFINITION NOT FOUND.}

\begin{itemize}
\item stacks work fine for when you just have flat functions, or even functions with functions inside of them 
but when you start returning functions it gets a bit messy
\item environments in languages with recursion but w/o functions returning functions are usually implemented with stacks
\end{itemize}

line (m, b) x = m * x + b

-- more explicit currying
line (m, b) = let 
     f y = m * y + b
     m f

line30 = line (3, 0)
line12 = line (1, 2)

line30 15 --what does the memory diagram look like here

\begin{itemize}
\item two ways of doing this
\begin{itemize}
\item line30 calls the compiler and you compile some code
\item a function has a pair of pointers
\begin{itemize}
\item a code pointer (points to code common to line30 and line12)
\item environment/closure pointer which points at some storage somewhere (probably the heap)
that has what we want
\end{itemize}
\end{itemize}
\end{itemize}
\section{lecture 12}
\label{sec-11}
\begin{itemize}
\item live
\begin{itemize}
\item data that will be used again
\item implies reachable
\item not computable
\end{itemize}
\item dead
\begin{itemize}
\item not that
\item not computable
\end{itemize}
\item reachable
\begin{itemize}
\item data you can find by following pointers starting from global and stack variables
\item computable
\end{itemize}
\item unreachable
\begin{itemize}
\item not that
\item implies dead
\item computable
\end{itemize}
\item tail position
\begin{itemize}
\item one chunk of sub-code is in tail position w.r.t. a surrounding piece of code if it will execute last
\end{itemize}
\item a tail-call is a call that is in tail position (usually w.r.t. some function)
\item at the point of a tail call the callee's environment s dead
\item tail optimization is the act of removing the callee's environment before starting a tail call
\item tail call optimization is just a call in a tail position, recursion isn't necessary
\begin{itemize}
\item important in recursion because that's where you're getting a lot of stack frames
\end{itemize}
\item 
\end{itemize}
\section{lecture 13 \textit{[2016-10-19 Wed]}}
\label{sec-12}
\begin{itemize}
\item strategies for making tail recursive
\begin{enumerate}
\item generalize
\item use accumilator
\item keep a list of work to do
\end{enumerate}
\item datatypes (haskel)
datatype Tree a = Empty
\begin{center}
\begin{tabular}{l}
Branch a (Tree a)(Tree a)\\
\end{tabular}
\end{center}
fred :: Tree Int
fred = Branch 3 (Branch 2 Empty Empty) Empty

 note: 
 fred =  3
        / $\backslash$
       / [empty]
      2
     / $\backslash$
    /   $\backslash$
   /     $\backslash$
[empty] [empty]

makeTree [] = Empty
makeTree (x:xs) = Branch x Empty (makeTree xs) -- a long scragly tree, all to the right

makeTree xs = let
         n = length xs
         m = (n-1) `div` 2
         ls = take m xs
         r:rs = drop m xs
         in Branch r (makeTree ls) (makeTree rs)

countNodes [] = 0
countNodes (Branch \_ l r) = 1 + (countNodes l) + (countNodes r)

countNode = helper 0 [] where
  helper acc list (Branch \_ l r) =
    helper (acc + 1) (r:list) l
  helper acc list Empty = dispatch acc list
    dispatch acc [] = acc
    dispatch acc (tree:work) = helper acc work tree
\end{itemize}
\section{lecture 15 \textit{[2016-10-31 Mon]}}
\label{sec-13}
\begin{itemize}
\item monads
\begin{itemize}
\item ?
\end{itemize}
\item persistent
\begin{itemize}
\item cant see it change
\item ownership issues are gone
\item data can be shared
\item copies can be shallow
\item equality can be reduced to pointer equality (with work)
\end{itemize}
\item ephemeral
\begin{itemize}
\item can see it change
\item some classical data stuctures (array) have no obvious persistent analog
\item classical algorithms need to be adapted to persistent data structures
\end{itemize}
\item persistent data structures allow us to think in new ways
\end{itemize}
\section{lecture 16 \textit{[2016-11-02 Wed]}}
\label{sec-14}
Queue:
data Queue a = Q[a][a][a] -- Q head, worker, tail
-- the length of the head == length of the worker + the length of tail
\section{lecture 17 \textit{[2016-11-04 Fri]}}
\label{sec-15}
\begin{itemize}
\item Monad
\begin{itemize}
\item has the following functions
return :: a -> m a -- takes a something to a container of somethings
(>>=) :: m a -> (a -> m b) -> m b
\item once you have a monad you can use do notation
\item lists as monads
return a = [a]
aList >>= f = concat(map f aList)
\item do
\begin{itemize}
\item do
a <- [10,20,30]
b <- [1,2]
return \$ a + b
\begin{itemize}
\item syntactic sugar for
[10,20,30] >>= ($\backslash$ a ->
  [1,2] >>= ($\backslash$ b -> [a+b]))
\end{itemize}
\end{itemize}
\end{itemize}
\end{itemize}
\subsection{{\bfseries\sffamily DONE} midterm exam}
\label{sec-15-1}
\begin{itemize}
\item recursion
\item tail position
\item tail call
\begin{itemize}
\item what's special about them?
\item garbage collection
\item how do they impact time and space
\end{itemize}
\item liveness
\item deadness
\item tail recursion
\begin{itemize}
\item how?
\begin{itemize}
\item generalize
\item list of sub problems
\item accumalator
\item continuations
\end{itemize}
\end{itemize}
\item continuations
\item static vs dynamic binding
\begin{itemize}
\item static means you look around the code
\item dynamic means you go up the call stack
\item how do scheme and haskell use by default (static)
\end{itemize}
\item paramertarize
\begin{itemize}
\item force dynamic
\end{itemize}
\item persistent vs ephemeral
\begin{itemize}
\item what is an example of a persistent Java class (will be on midterm)
\begin{itemize}
\item String
\end{itemize}
\item how would you make a persistent queue in Haskell
\begin{itemize}
\item use at least two lists, a head and a reverse tail
\end{itemize}
\end{itemize}
\item haskell type classes
\begin{itemize}
\item how to define
\item instances
\item what is a Monad
\begin{itemize}
\item why is it special
\item desugar a do block
\end{itemize}
\end{itemize}
\end{itemize}
\section{lecture 18 \textit{[2016-11-09 Wed]}}
\label{sec-16}
\begin{itemize}
\item 
\end{itemize}
\section{lecture 19 \textit{[2016-11-14 Mon]}}
\label{sec-17}
\begin{itemize}
\item haskell
\begin{itemize}
\item f x = 3 * x + 1
\end{itemize}
\item prolog
\begin{itemize}
\item f(X, Y) :- Y is 3 * X + 1.
\begin{itemize}
\item uppercase is a variable
\item is operator numerically evaluates its righthand side then the expression is true if both sides are equal
\item syntactic sugar for '+'('*'(3,x),1)
\end{itemize}
\item prolog is relational not functional
\item ?- f(3,11).
\begin{itemize}
\item False
\end{itemize}
\item ?- f(3, 10).
\begin{itemize}
\item True
\end{itemize}
\item ?- f(3, A).
\begin{itemize}
\item A = 10
\end{itemize}
\item f(X, 10).
\begin{itemize}
\item an error
\end{itemize}
\item ?- append(X, [4,5,6], [1,2,3,4,5,6]).
\begin{itemize}
\item X = [1,2,3]
\end{itemize}
\item ?- append(X, X, [1,2,3,4,5,6]).
\begin{itemize}
\item False
\end{itemize}
\item ?- append(X, Y, [1,2,3,4,5,6]).
\begin{itemize}
\item X = [], Y = [1..6];
\end{itemize}
\item n1.
\begin{itemize}
\item is just a fact which is equivalent to ..
\begin{itemize}
\item n1().
\end{itemize}
\end{itemize}
\item fat(cat, fred(2, 3+1))
\begin{itemize}
\item an assertion
\end{itemize}
\item "think of prolog like partially digested natural language"
\item 
\item directlyNorthOf(pg, redRock).
\item directlyNorthOf(redRock, stoner).
\item directlyNorthOf(stoner, hixon).
\item directlyNorthOf(hixon, quesnel)
\item northOf(X, Y) :- directlyNorthOf(X, Y).
\item northOf(X, Y) :- directlyNorthOf(X, Z), northOf(Z, Y).
\item 
\item prolog is all about universal horn clauses, where a universal horn clause is a universally 
quantified disjunction of atomic clauses, all but one of which is negated
\item practical notes
\begin{itemize}
\item ?- is the prompt in the interpreter
\item ?- consult("filename.pl"). will read a file
\item ?- make. will reconsult anything that's been consulted in this session
\item ?- help(help). should give you a gui based browser
\end{itemize}
\end{itemize}
\end{itemize}
\section{lecture 20 \textit{[2016-11-16 Wed]}}
\label{sec-18}
\begin{itemize}
\item prolog syntax
\begin{itemize}
\item comments
\begin{itemize}
\item \emph{\textbf{block comments}}
\item \% line comments
\end{itemize}
\item strings are "double quoted" with $\backslash$ escapes
\item `backticks are a string made of a list of chars'
\item Variables start with Capitals
\item \_ is a wildcard
\item atoms start with lowercase letters or can be quoted with 'single quotes'
\item lists are built using [] (empty) and [.|.] (cons) with lots of sugar support
\begin{itemize}
\item ex: [1,2,3] = [1 | [2 | [3 | [] ] ] ]
\end{itemize}
\end{itemize}
\item prolog append
append([], Ys, Ys).
append([X|Xs], Ys, [X|Zs]) :- append(Xs, Ys, Zs).
\item 
\item λ-calculus
\begin{itemize}
\item (λ (x) (λ (y) (λ (z) \ldots{} )))
\item λx.(x is bound in here)
\item the pure λ-calculus is built from variables, abstractions and applications f(x)
\item example
\begin{itemize}
\item flip f x y = f y x
\item λfxy.[f(y)](x)
\end{itemize}
\end{itemize}
\end{itemize}
\section{lecture 21 \textit{[2016-11-19 Sat]}}
\label{sec-19}
\begin{itemize}
\item α-conversion
\begin{itemize}
\item if you change the name of a bound variable it should not change the meaning
\begin{itemize}
\item λx.x should be the same as λy.y
\end{itemize}
\end{itemize}
\item β-reduction is how the λ-calculus "works"
\begin{itemize}
\item (λx.T)(A) \textasciitilde{}~\textasciitilde{}~\textasciitilde{}> T[x/A]
\begin{itemize}
\item replace each x in T with A
\begin{itemize}
\item\relax [λx.sin(x)](3) \textasciitilde{}~\textasciitilde{}~\textasciitilde{}~\textasciitilde{}> sin(3)
\item\relax [λx.(sin(x) + (λx.x)(5))](3)
\end{itemize}
\end{itemize}
\end{itemize}
\item de Bruijn indices
\begin{itemize}
\item look it up, seems important
\end{itemize}
\item how do we represent λ-calculus in Prolog?
\begin{itemize}
\item unbd(sin) unbound
\item app(unbd(sin),unbd(3)) apply
\item dB(3) de Bruijn indices
\item abstractions
\begin{itemize}
\item abs(x, term) where x = string, term = λ-term
\item abs0(x, term) where x is a suggested variable name, term uses dB(N) for x
\end{itemize}
\end{itemize}
\item natural numbers in λ-calc
\begin{itemize}
\item two is λfx.f(f(x))
\item abs("f", abs0("x", app(dB(1), app(dB(1), dB(0)))))
\end{itemize}
\item absToAbs0(Term1, Term2) :- absToAbs0(Term1, Term2, []).
absToAbs0(app(F,X), app(F1, X1), List) :- 
  absToAbs0(F, F1, List),
  absToAbs0(x, X1, List).
absToAbs0(unbd(X), unbd(X), \_).
absToAbs0(Q, dB(N), List) :-
  isVar(Q),
  nth0(N, List, Q).
absToAbs0(abs(X, Term), abs0(X, Term1), List) :-
  absToAbs0(Term, Term1, [X|List]).
\end{itemize}
\section{lecture 22 \textit{[2016-11-21 Mon]}}
\label{sec-20}
\subsection{lambda calc turing machine}
\label{sec-20-1}
T : λxy.x
F : λxy.y
if : λxyz.xyz
and : λpq.if(p)(q)(F)

pair : λab.λc.if(c)(a)(b)
first : λp.p(T)
second : λp.p(F)

from now on pairs are
<a, b> = pair(a)(b)
p1 = first(p)

empty : <F,F>
cons : λab.<T,<a,B>>
empty? : λd.not(d1)
cons? : λd.d1
car : λd.(d2)1 
cdr : λd.(d2)2

\subsection{numbers (do a thing n times)}
\label{sec-20-2}
0: λfx.x
1: λfx.f(x)
2: λfx.f(f(x))

zero: λN.N(λx.F)(T)
add1: λNfx:Nf(f(x))

operators
+: λmn.λfx.m(f)(nfx)
*: λmn.λfx.m(nf)x






\subsection{recursion}
\label{sec-20-3}
how to do recursion without recursive function definitions. the μ (or Y) combinator

length [] = 0
length (\_:xs) = 1 + length xs

ellImprov f [] = 0
ellImprov f (\_:xs) = 1 + f xs

ellImprov length = length
ellImprov undefined = \{[] -> 0, otherwise -> undefined\}

ellImprov ellImprov (ellImprov undefined) = \{correct for length <= 1, otherwise undefined\}
allImprov$^{\text{(1000)}}$ undefinied = \{correct up to length 999\}

μ gg x = gg (μ gg)x

\section{lecture 23 \textit{[2016-11-25 Fri]}}
\label{sec-21}
\begin{itemize}
\item DCCs are syntactic sugar
goal(fred(A,B,C))
          -->
           matchA(A), matchB(B), MatchC(C)

goal(fred(A,B,C), In, Out) :-
  matchA(A, In, Out1),
  matchB(B, Out1, Out2),
  matchC(C, out2, Out).

goal(x, "aabbbc!" out)
x = fred("aa","bbb","c")
out = "!"
\end{itemize}
\section{lecture 25 \& 26 \textit{[2016-11-28 Mon]}}
\label{sec-22}
\begin{itemize}
\item program
pred
pred
\item bindings
goal
\begin{center}
\begin{tabular}{}
\\
\end{tabular}
\end{center}
  v
goal 2
\item backtrack stack
\end{itemize}
\begin{center}
\begin{tabular}{l}
state of the machine at some point\\
state of the machine at some point\\
\end{tabular}
\end{center}
\begin{itemize}
\item if you run out of goals you succeed
\item not matching a goal causes backtracking
\item attempting to backtrack when backtrack stack is empty results in failure
\item when a goal matches the head of a rule
\begin{enumerate}
\item add unification bindings
\item replace the goal with the RHS of the rule
\begin{itemize}
\item depth first search
\end{itemize}
\item possibly add an entry to backtrack stack
\end{enumerate}
\item the cut (!) 0 arg predicate
\begin{itemize}
\item always succeeds
\item removes choice points introduced since matching the head of the current rule
\item is useful if we know some choices will fail
\item also useful for forcing flow
\end{itemize}
myNot(Goal) :- once(Goal), !, fail.
myNot(Goal).
\end{itemize}

\subsection{unification}
\label{sec-22-1}
\begin{itemize}
\item if we have two terms with variables, a unifier is a list of variable assignments that make the terms identical
cat(house, bat(X), U)
cat(V, Y,dog(V))
is there a list of assignments for U,V,X,Y that will make those terms identical
V = house
U = dog(house)
Y = bat(X)
\item variable
\begin{itemize}
\item if the variable is already bound, replace it with the binding
\item if the variable isn't bound, bind it to the other term
\begin{itemize}
\item ex. if you've got an unbound u and you are matching it to "cat" then add u="cat" to the unifier
\item it matters operationally which variable to bind if unifying two unbound variables
\begin{itemize}
\item THE OCCURS CHECK
\end{itemize}
\end{itemize}
\end{itemize}
\item Atoms
\begin{itemize}
\item only unify with themselves
\end{itemize}
\item Terms
\begin{itemize}
\item only unify with terms
\item the heads and arities must match
\begin{itemize}
\item app(a,b,c) wont match app(a,b)
\end{itemize}
\item the arguments must pairwise unify
\begin{itemize}
\item app(a,b,X) would match app(X,b,Y)
\begin{itemize}
\item X = a, Y = a
\end{itemize}
\end{itemize}
\end{itemize}
\end{itemize}

\subsection{metalogical predicates}
\label{sec-22-2}
\begin{verbatim}
plus(X,Y,Z) :- var(Y), !, Y is Z - X
plus(X,Y,Z) :- var(X), !, X is Z - Y
plus(X,Y,Z) :- var(Z), !, Z is X + Y
\end{verbatim}

\begin{verbatim}
plus(X,Y,Z) :- Z is X+Y
\end{verbatim}
is a metalogical prediacate, var, nonvar, ground

\begin{itemize}
\item type predicates
\begin{itemize}
\item atom
\item atomic
\item number
\end{itemize}
\end{itemize}

integer(X), is true if X is an integer

\subsection{control predicates}
\label{sec-22-3}
!, not, fail/false, true, repeat

\subsection{a final exam question}
\label{sec-22-4}
given two terms do they unify?\\
ex:\\
fred(cat(z), w)\\
fred(cat(w), y)\\
w = z, y = z\\

\section{review}
\label{sec-23}
\begin{itemize}
\item product of a list
\end{itemize}
\begin{verbatim}
product([], 1).
product([X|Xs], Y) :- product(Xs, Z),
		      Y is X * Z.
\end{verbatim}
\begin{verbatim}
product = prod 1 where 
       prod acc [] = acc
       prod acc (x:xs) = prod (acc*x) xs
\end{verbatim}
\begin{itemize}
\item possible test question: reverse list in prolog
\item course overview
\begin{itemize}
\item memorization trick
\begin{itemize}
\item shrink things down
\begin{itemize}
\item start by categorizing
\item then combine categories
\item then merge into one word
\item then remember that word and unpack it in your head when neccesary
\end{itemize}
\end{itemize}
\end{itemize}
\item types of language
\begin{itemize}
\item static vs dynamic
\item strict vs lazy
\begin{itemize}
\item strict
\begin{itemize}
\item scheme
\item prolog, unification happens on the head before you start working on the body
\end{itemize}
\item lazy
\begin{itemize}
\item haskell
\end{itemize}
\end{itemize}
\end{itemize}
\item do notation / monads
\begin{itemize}
\item probably not too much
\end{itemize}
\item underlying math
\begin{itemize}
\item whats a relation
\item whats a function
\item what's a partial function
\begin{itemize}
\item how to count partial functions
\end{itemize}
\item what's a maybe type good for
\begin{itemize}
\item good way to represent partial functions
\end{itemize}
\item some form of translation question, translate a prolog statement to 141 math
\end{itemize}
\item prolog unification questions
\end{itemize}
% Emacs 25.1.1 (Org mode 8.2.10)
\end{document}