\documentclass{article}
\begin{document}
\title{CPSC 300 Project Report}
\author{Barry Warnock\\
  230113824}
\date{December 5, 2016}
\maketitle


\section*{Introduction}
For this project we were asked to set up a web interface with a nodeJs backend that would read and graph data from
a NIRS (Near-infrared spectroscopy) sensor. These sensors are used to detect blood flow by bouncing near infrared light
through the skin and measuring how much it can read through the skin. we opted to drop the nodeJs in favour of making
it a chrome web app to simplify both development as well as set up for the end user.

\section*{Contribution}
We broke down the project into three parts, front-end, device through USB, and device through Bluetooth. I was assigned to make the device
work through Bluetooth. A point by point high level breakdown of my contribution is as follows:
\begin{itemize}
\item after Jeremy discovered that Chrome might have the functionality we need I managed to find the specific parts of the API
  that we needed
\item I along with Adam and Jeremy planned out a data structure to use for the device data
\item Me and Adam experimented with the device to figure out how it and the API worked, this led me to discover that
  the API we really wanted was chrome.serial which would allow us to treat USB and Bluetooth the same way
\item I wrote code that was able to detect the device but was unable to read from it
\end{itemize}
Unfortunately we were unable to get the device to work, our theory is that it was waiting for some byte-sequence
before it would stop sending data. We did end up getting the source code to the program that is currently being used to get the
NIRS data but it was too late.

\section*{Feedback}
Overall I would say that we were absolutely asked to use the wrong technology, JavaScript is fine for web scripting in a web page but for
a program that needs to interact with external devices where the user wants the ui to react like a desktop application on a windows computer there
are much better decisions (C\#, Java, Python). Additionally if we were to have made this a client-server application not only would this
have made development more complex, it would have also required anybody who wanted to use it to set up a NodeJs stack on their device.

As far as the actual engineering section goes I think it may have helped to have access to the software we were meant to replace from the start.
It would have helped show us what functionality we need as well as see how the device works. 

\end{document}