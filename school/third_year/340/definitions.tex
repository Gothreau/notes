% Created 2016-12-12 Mon 13:32
\documentclass[11pt]{article}
\usepackage[utf8]{inputenc}
\usepackage[T1]{fontenc}
\usepackage{fixltx2e}
\usepackage{graphicx}
\usepackage{longtable}
\usepackage{float}
\usepackage{wrapfig}
\usepackage{rotating}
\usepackage[normalem]{ulem}
\usepackage{amsmath}
\usepackage{textcomp}
\usepackage{marvosym}
\usepackage{wasysym}
\usepackage{amssymb}
\usepackage{hyperref}
\tolerance=1000
\author{Barry }
\date{\today}
\title{definitions}
\hypersetup{
  pdfkeywords={},
  pdfsubject={},
  pdfcreator={Emacs 25.1.1 (Org mode 8.2.10)}}
\begin{document}

\maketitle
\tableofcontents

\section{Definitions}
\label{sec-1}
\subsection{DFA:}
\label{sec-1-1}
DFA M = (Q,$\Sigma$,$\delta$,q$_{\text{0}}$,F) consisting of
\begin{itemize}
\item a finite set of states (Q)
\item a finite set of input symbols called the alphabet ($\Sigma$)
\item a transition function ($\delta$: Q x $\Sigma$ $\rightarrow$ Q)
\item an initial or start state (q$_{\text{0}}$ $\in$ Q)
\item a set of accepting states (F \subseteq Q)
\end{itemize}

a string w over the alphabet $\Sigma$ is accepted by M if a sequence of states, r$_{\text{0}}$,r$_{\text{1}}$,\ldots{},r$_{\text{n}}$ exists in Q with the following conditions:
\begin{center}
\begin{enumerate}
\item r$_{\text{0}}$ = q$_{\text{0}}$
\item r$_{\text{i+1}}$ = $\delta$(r$_{\text{j}}$, a$_{\text{i+1}}$), for i = 0,\ldots{},n-1
\item r$_{\text{n}}$ $\in$ F
\end{enumerate}
\end{center}

\subsection{NFA:}
\label{sec-1-2}
NFA M = (Q,$\Sigma$,$\delta$,q$_{\text{0}}$,F) consisting of
\begin{itemize}
\item a finite set of states (Q)
\item a finite set of input symbols called the alphabet ($\Sigma$)
\item a transition function ($\delta$: Q x $\Sigma$ $\rightarrow$ 2$^{\text{Q}}$)
\item an initial or start state (q$_{\text{0}}$ $\in$ Q)
\item a set of accepting states (F \subseteq Q)
\end{itemize}

The same rules govern a word w's acceptance by M as with a DFA

\subsection{$\epsilon$-NFA:}
\label{sec-1-3}
An $\epsilon$-NFA M = (Q,Sigma,$\delta$,q$_{\text{0}}$,F) consisting of
\begin{itemize}
\item a finite set of states (Q)
\item a finite set of input symbols called the alphabet ($\Sigma$)
\item a transition function ($\delta$: Q x ($\Sigma$ $\cup$ $\epsilon$) $\rightarrow$ 2$^{\text{Q}}$)
\item an initial or start state (q$_{\text{0}}$ $\in$ Q)
\item a set of accepting states (F \subseteq Q)
\end{itemize}

The same rules govern a word w's acceptance by M as with a DFA

\subsection{Grammar:}
\label{sec-1-4}
A grammar G = (N,$\Sigma$,P,S) where
\begin{itemize}
\item a finite set of nonterminal symbols that is disjoint with the strings formed by G (N)
\item a finite set of terminal symbols that is disjoint from N ($\Sigma$)
\item a finite set of production rules (P) each of the form:
\end{itemize}
\begin{center}
($\Sigma$ $\cup$ N)$^{\text{*}}$ N($\Sigma$ $\cup$ N)$^{\text{*}}$ $\rightarrow$ ($\Sigma$ $\cup$ N)$^{\text{*}}$
\end{center}
\begin{itemize}
\item a distinguished symbol S $\in$ N that is the start symbol
\end{itemize}

\subsubsection{Right Linear}
\label{sec-1-4-1}
All nonterminals in the righthand side of productions are at the right ends
\subsubsection{Left Linear}
\label{sec-1-4-2}
All nonterminals in the righthand side of productions are at the left ends
\subsubsection{Linear Grammar}
\label{sec-1-4-3}
All of the productions in a linear grammar are right or left linear (not neccesarily all the same)
\subsubsection{Regular Grammar}
\label{sec-1-4-4}
A left or right linear grammar
\subsubsection{context free}
\label{sec-1-4-5}
a context free grammar imposes the following rules on elements of P
\begin{center}
A $\rightarrow$ x where A $\in$ V , x $\in$ (V U T)$^{\text{*}}$
\end{center}

\subsection{Pumping Lemma for Regular languages:}
\label{sec-1-5}
let L be a regular language. then there exists a constant n $\ge$ 1 st. if w is any string in L st. |w| $\ge$ n, we can find x,y,z st. w = xyz
\begin{enumerate}
\item |xy| $\le$ n
\item y $\ne$ $\epsilon$
\item for all k $\ge$ 0, xy$^{\text{k}}$z $\in$ L
\end{enumerate}

\subsection{pumping lemma for Context free languages}
\label{sec-1-6}
let L be a context free language. then there exists a constant n $\ge$ 1 st. if z is any string in L st. |z| $\ge$ n, we can find u,v,w,x,y st. z = uvwxy
\begin{enumerate}
\item |vwx| $\le$ n
\item vx $\ne$ $\epsilon$
\item for all i $\ge$ 0, uv$^{\text{i}}$wx$^{\text{i}}$y $\in$ L
\end{enumerate}

\subsection{PDA}
\label{sec-1-7}
PDA M = (Q,$\Sigma$,$\Gamma$,$\delta$,Z,F) consisting of:
\begin{itemize}
\item a finite set of states (Q)
\item a finite set of input symbols called the alphabet ($\Sigma$)
\item a finite set of stack symbols called the stack alphabet ($\Gamma$)
\item a transition relation $\delta$ \subseteq Q x ($\Sigma$ $\cup$ \{$\epsilon$\}) x $\Gamma$ x Q x $\Gamma$$^{\text{*}}$
\item an initial or start state (q$_{\text{0}}$ $\in$ Q)
\item Z $\in$ $\Gamma$ is the inital stack symbol
\item a set of accepting states (F \subseteq Q)
\end{itemize}

An element (p,a,A,q,$\alpha$) $\in$ $\delta$ is a transition of M. It has the intended meaning that M, in state p $\in$ Q, on the input 
a $\in$ $\Sigma$ $\cup$ \{$\epsilon$\} and with A $\in$ $\Gamma$ as topmost stack symbol, may read a, change the state to q, pop A, replacing it
by pushing $\alpha$ $\in$ $\Gamma$$^{\text{*}}$. 

\subsection{Turing Machine}
\label{sec-1-8}
A Turing Machine M = (Q,$\Gamma$,b,$\Sigma$,$\delta$,q$_{\text{0}}$,F) consisting of:
\begin{itemize}
\item a finite set of states (Q)
\item a finite, non-empty set of tape alphabet symbols ($\Gamma$)
\item a blank symbol, the only symbol allowed to occur on the tape infinitely often at any step (b $\in$ $\Gamma$)
\item a set of input symbols ($\Sigma$ \subseteq $\Gamma$ - \{b\})
\item a partial function $\delta$: Q x $\Gamma$ $\rightarrow$ Q x $\Gamma$ x \{L,R\}
\item an initial state (q$_{\text{0}}$)
\item a set of final or accepting states (F \subseteq Q)
\end{itemize}
\subsubsection{Turing Machine ID}
\label{sec-1-8-1}
the ID of a turing machine = $\alpha$qB where $\alpha$ is the symbols to the left, q is the current state, B 
is the curent symbol and the everything to the right.

\subsection{Recursivly Enumerable Language}
\label{sec-1-9}
a recursivly enumerable language is computable by a turing machine and will halt if there is an answer
\subsection{Recursive Language}
\label{sec-1-10}
a recursive language is computable by a turing machine and the machine will eventually halt
\subsection{Chomsky hierarchy}
\label{sec-1-11}
each level contains the levels below it
\begin{itemize}
\item Recursively Enumerable
\begin{itemize}
\item Turing Machines
\end{itemize}
\item Context Sensitive
\begin{itemize}
\item Linear Bounded Turing Machine
\end{itemize}
\item Context Free
\begin{itemize}
\item Context Free Grammar
\item Pushdown Automata
\end{itemize}
\item Regular
\begin{itemize}
\item Regular Expression
\item DFA
\item NFA
\item $\epsilon$-NFA
\item Regular Grammar
\end{itemize}
\end{itemize}
% Emacs 25.1.1 (Org mode 8.2.10)
\end{document}