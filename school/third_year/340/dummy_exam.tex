\documentclass{exam}
\runningfooter{CPSC 340}{Practice Exam}{Page \thepage\ of \numpages}
\runningfootrule
\begin{document}
\title{CPSC 340 Practice Exam}
\date{}
\maketitle
\vspace{0.1in}
\makebox[\textwidth]{Name and section:\enspace\hrulefill}
\vspace{0.2in}
\makebox[\textwidth]{Instructor’s name:\enspace\hrulefill}
\vspace{1cm}

\section*{Problems From Homework}
\begin{questions}
  \question
  Give DFA's accepting the following languages over the alphabet {0,1}
  \begin{parts}
    \part
    The set of all strings ending in 00.
    \part
    The set of all strings with three consecutive 0's.
    \part
    the set of strings with 011 as a sub string.
  \end{parts}

  \question
  Let A be a DFA and \textit{a} a particular input symbol of A, such
  that for all states \textit{a} of A we have $\delta(q, a) = q$.
  \begin{parts}
    \part
    Show by induction on n that for all $n \ge 0, \delta(q,a^n) = 1$,
    where $a^n$ is the string consisting of n a's.
    \part
    Show that either ${a}^* \subseteq L(A) or {a}^a \cap L(A) =
    \emptyset$ 
  \end{parts}

  \question
  Consider the DFA with the following transition table:
  \begin{center}
    \begin{tabular}{c|c|c}
      &0&1\\
      \hline
      $\rightarrow A$&A&B\\
      $*B$&B&A
    \end{tabular}
  \end{center}
  Informally describe the language accepted by this DFA and prove by
  induction on the length of an input string that your description is
  correct Hint: When setting up the inductive hypothesis it is wise
  to make a statement about what inputs get you to each state, not just
  what inputs get you to the accepting state.

  \question
  write a program to solve the following problem:\\
  given an NFA $M = (Q, \Sigma, \delta, q0, F)$ and an input string of
  M w, you are able to determine if w is in L(M)

  \question
  Convert to a DFA the following NFA:
  \begin{center}
    \begin{tabular}{c|c|c}
      $\rightarrow p$&{p,q}&{p}\\
      \hline
      q&{r}&{r}\\
      r&{s}&$\emptyset$\\
      $*s$&{s}&{s}
    \end{tabular}
  \end{center}

  \question
  Give a nondeterministic finite automata to accept the following
  language. Try to take advantage of nondeterminism as much as possible.
  \begin{center}
    The set of strings over the alphabet {0,1,..,9} such that the final
    digit has appeared before.
  \end{center}

  \question
  Design NFA's to recognize the following sets of stings.
  \begin{parts}
    \part
    abc, abd, and aacd. Assume the alphabet is {a,b,c,d}.
    \part
    0101, 101, 011.
    \part
    ab, bc, and ca. 
  \end{parts}

  \question
  Convert each of your NFA's from question 7 to DFA's

  \question
  Here is a transition table for a DFA:
  \begin{center}
    \begin{tabular}{c|c|c}
      $\rightarrow q1$
    \end{tabular}
  \end{center}

\end{questions}
\end{document}